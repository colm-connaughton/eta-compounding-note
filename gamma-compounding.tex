\documentclass[11pt]{article}
\usepackage[T1]{fontenc}
\usepackage[english]{babel}
\usepackage{graphicx}
\usepackage{palatino}
\usepackage{helvet}
\usepackage{times}
\usepackage{layout}
\usepackage[a4paper,top=2.0cm, right=2.0cm, bottom=2.0cm, left=2.0cm]{geometry}
\usepackage{enumitem}
\usepackage{amsthm}
\usepackage{url}
\usepackage{multicol,caption}

\setlist{nolistsep}

\usepackage{fancyhdr}
\pagestyle{fancy}
\lhead{{\hvnb Colm}}
\chead{}
\rhead{}
\lfoot{}
\cfoot{}
\rfoot{}

\newcommand*{\helvetica}{\fontfamily{phv}\selectfont}
\newcommand*{\helveticanarrow}{\fontfamily{phv}\fontseries{mc}\selectfont}
\newcommand*{\hvnb}{\fontfamily{phv}\fontseries{bc}\selectfont}
\newcommand*{\palatino}{\fontfamily{ppl}\selectfont}
\newcommand*{\timesroman}{\fontfamily{ptm}\selectfont}

% Commands to produce formatted layout
\newcommand*{\projecttitle}[1]{\begin{center}\Large\hvnb{\color{blue} #1}\end{center}}
\newcommand*{\theauthor}[1]{\noindent  \helveticanarrow{#1}\\}



% Support for multiple bibliographies
\usepackage[sectionbib,numbers]{natbib}
\usepackage{chapterbib}

\usepackage[compact]{titlesec}
\usepackage{lipsum}
\usepackage{xcolor}
\usepackage{amsmath}
\usepackage{amsfonts}

\DeclareMathOperator*{\argmax}{arg\,max}
\DeclareMathOperator*{\argmin}{arg\,min}

\titleformat{\section}
  {\large\hvnb\color{blue}}{\thesection}{1em}{}
\titleformat{\subsection}
  {\hvnb}{\thesubsection}{1em}{}
\titleformat{\chapter}
  {\Large\hvnb\color{blue}}{Lecture \thechapter:}{1em}{}

\newenvironment{Figure} 
  {\par\medskip\noindent\minipage{\linewidth}}
  {\endminipage\par\medskip}

\newcommand{\pd}[2]{\frac{\partial #1}{\partial #2}}
\newcommand{\dd}[2]{\frac{d {#1}}{d {#2}}}
\DeclareMathOperator{\sgn}{sgn}

\begin{document}
\setlength{\bibsep}{0.2pt}
\setlength{\itemsep}{0.2pt}

\helveticanarrow
\rhead{\hvnb  29/06/2023 LML working notes}

\projecttitle{The $\gamma$-compounding random walk}

%\theauthor{Colm Connaughton} 


\begin{multicols}{2}

%\section{Definition of the model}
\cite{redner1990random} \cite{carr2022generalized}

Define the generalised exponential and logarithm as
\begin{align}
\label{eq:exp_gamma}
\exp_\gamma (x) = & \left\{
\begin{array}{ll} 
\left( 1 + \gamma\,x\right)^\frac{1}{\gamma} & \text{$0 < \gamma \leq 1$}\\
\exp(x) & \text{$\gamma=0$}
\end{array}
\right. \\
\label{eq:log_gamma}
\log_\gamma (x) = & \left\{
\begin{array}{ll} 
\frac{1}{\gamma}\left( x^\gamma -1 \right) &  \text{$0 < \gamma \leq 1$}\\
\log(x) & \text{$\gamma=0$}
\end{array}
\right. ,
\end{align}
and the generalised compounding operator, $\otimes$, as
\begin{align}
x \otimes y = & \exp_\gamma\left[ \log_\gamma(x )+ \log_\gamma(y)\right].
\end{align}
We are interested in studying the gamma-compounding random walk in discrete time with growth factors $g+r$ and $g-r$ occurring with equal probability: 
\begin{align}
\label{eq-multRandomWalk}
x_{t+1} = \left\{ 
\begin{array}{ll}
x_t \otimes \left(g +r \right) & \text{with probability $\frac{1}{2}$}\\
x_t \otimes \left(g - r \right)  & \text{with probability $\frac{1}{2}$}.
\end{array}
\right.
\end{align}
with $x_0 = X_0$.

If, after playing $T$ rounds of the game, we experience $n$ ``wins" (and $T-n$ "losses"), then $x_T$ will take the value
\begin{align*}
x_T =  &X_0\otimes \underbrace{(g+r)\otimes \ldots \otimes (g+r)}_{n\text{-times}} \otimes \underbrace{(g-r)\otimes \ldots \otimes (g-r)}_{T-n\text{-times}} \\
=& X_0\otimes\exp_\gamma\left[n\,\log_\gamma(g+r)\right] \otimes \exp_\gamma\left[(T-n)\,\log_\gamma(g-r) \right]\\
=& X_0\otimes \exp_\gamma\left[ n\,\log_\gamma(g+r) + (T-n)\log_\gamma(g-r) \right].
\end{align*}
The probability of this value is
\begin{align}
\label{eq:binomialDistr}
p(n) = {T \choose n} \left(\frac{1}{2}\right)^T,
\end{align}
where ${T \choose n}$ is the binomial coefficient -  the number of ways in which $n$ wins can occur in a sequence of $T$ rounds of the game.
The expectation value of $x_T$ is therefore
\begin{align}
\label{eq-expectationxT}\mathbb{E}\left[x_T \right] =& \sum_{n=0}^T  {T \choose n} \left(\frac{1}{2}\right)^T  X_0\otimes\exp_\gamma\left[ n\,\log_\gamma(g+r) \right.\\
\nonumber & \left. + (T-n)\log_\gamma(g-r) \right].
\end{align}
This sum can be done exactly at least for the case $\gamma = \frac{1}{2}$ (details later):
\begin{align}
x_T =& \left(\sqrt{X_0}-1 +\sqrt{P(T, \sqrt{g+r}, \sqrt{g-r})} \right)^2,
\end{align}
where
\begin{align*}
P(n, x, y) =& \frac{1}{4} n (n+1) x^2+\frac{1}{2} n (n-1) x y\\
& -n (n-1) x+\frac{1}{4} n (n+1) y^2-n (n-1) y\\
& +(n-1)^2.
\end{align*}
\bibliographystyle{plain}
\bibliography{refs}


\end{multicols}
\end{document}
